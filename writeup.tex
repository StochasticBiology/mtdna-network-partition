\documentclass{article}

\usepackage{amsmath}
\usepackage{graphicx}
\usepackage[a4paper, portrait, margin=2cm]{geometry}

\newcommand{\be}{\begin{equation}}
\newcommand{\ee}{\end{equation}}

\title{Mitochondrial network structure modulates variance in mtDNA copy number and heteroplasmy generated by cell divisions}
\author{Robert C. Glastad, Iain G. Johnston}
\date{}

\begin{document}

\maketitle

Inheritance of mtDNA is of central importance in eukaryotic biology.

Most quantitative models of mtDNA partitioning at cell divisions consider either independent mtDNA molecules or clusters, without spatial embedding. However, there is growing recognition that the physical structure of mitochondria affects the genetic dynamics of mtDNA populations. Here we investigated how the inclusion of mtDNA in a mitochondrial network influences mtDNA inheritance at cell divisions.



\subsection*{Network structure and mtDNA inclusion shape copy number and heteroplasmy variance from cell divisions}

To build intuition about the influence of network structure on mtDNA inheritance, we first considered a simple computational model for the spatial distribution of mitochondria and mtDNAs within a cell. We simulate a random mitochondrial network structure via an elongation and branching process, initiated at a number of evenly-spaced seed points around a circular 2D cell. Network segments elongate and branch according to Poissonian dynamics, and terminate if they impact the cell boundary. Network growth is simulated until a given mass is created; if all segments terminate before this mass is reached, the algorithm is re-seeded. This process gives rise to model network structures distributed with tunable uniformity throughout the cell: a high number of seed points gives homogeneous structure, and a low number of seed points often leads to network localisation in a subset of cellular locations (Fig. \ref{fig-snaps}).

We then place mtDNAs in cellular positions shaped by this model network structure. We consider two mtDNA types: wildtype $w$ and mutant $m$. Proportions $p$ of wildtype and $q$ of mutant mtDNA are placed in the constructed network; the remaining mtDNAs are placed randomly and uniformly throughout the cell. At first, the networked mtDNAs are placed randomly and uniformly throughout the network: we will later consider self-avoidance. We also later consider perturbations to these mtDNA positions.

\begin{figure}
  \includegraphics[width=\textwidth]{plotcell.png}
  \caption{Cell snapshots from individual simulations. Network in grey, cell boundary in black, wildtype mtDNA in red, mutant in blue.}
  \label{fig-snaps}
\end{figure}

Finally, we partition the cell, and record the number of wildtype and mutant mtDNAs is one daughter, as well as the heteroplasmy $h = m/(w+m)$. Partitioning occurs via a division plane, which assigns a proportion $p_c$ of the original cell volume to the daughter of interest. When $p_c = 1/2$, division is symmetric. We are particularly interested in the heteroplasmy variance $V(h)$ and copy number variance $V(N)$ across many realisations of this system. Under a `null' case of no network structure, partitioning of mtDNAs is binomial, predicting 

\begin{eqnarray*} 
V(N) = N_0 p_c(1-p_c) \\
V(h) = h(1-h) / N_0 XXX
\end{eqnarray*}

Simulation results are shown in Fig. \ref{fig1a}, plotted relative to this null case. 

\begin{figure}
  \includegraphics[width=\textwidth]{symm-both-stats-no-repel.png}
  \caption{Partitioning stats. Top left, simulated $V(h)$ for different numbers of seeds and $h$. Top left, simulated $V(N)$. Bottom left, predicted $V(h)$. Bottom right, predicted $V(N)$.}
  \label{fig1a}
\end{figure}

\subsection*{Genetic bias in mtDNA network inclusion maximises heteroplasmy variance} 
There is a clear pattern that $V(h)$ takes minimum values when $p=q$, that is when network inclusion probabilities are equal for mutant and wildtype mtDNA. When the two differ, $V(h)$ increases, with highest values occurring when the majority mtDNA type is exclusively contained in the network and the minority type exclusively in the cytoplasm. 

This result may seem counterintuitive at first glance: when the network is highly heterogeneous, it might be expected that including all mtDNAs there would maximise variance. This is true for copy number variance, but not for heteroplasmy variance, because including both types equally induces correlation in their inheritance and lowers variance. The maximum $V(h)$ is achieved by embedding the majority type in the high-variance network and having the minority type in the uncorrelated cytoplasm. 

\subsection*{Analytical results}
We next sought an analytic framework with which to understand these behaviours. We proposed the following model for the simple case without self-avoidance or diffusion:

\begin{eqnarray}
  U & \sim & Beta(\alpha, \beta) \\
  W_n & \sim & Bin(p(1-h_0)N_0, U) \\
  M_n & \sim & Bin(qh_0 N_0, U) \\
  W_c & \sim & Bin((1-p)(1-h_0)N_0, \rho) \\
  M_c & \sim & Bin((1-q)h_0 N_0, \rho).
\end{eqnarray}

Here, a daughter cell inherits a beta-distributed amount of mitochondrial network mass $U$. The $p(1-h)N$ networked wildtype mtDNAs and the $qhN$ networked mutant mtDNAs are inherited by this daughter with probability $U$, modelling their presence within the proportion of the network it inherits. The remaining unnetworked mtDNAs are inherited binomially with probability $\rho$. $\rho$ and the mean $\alpha/(\alpha+\beta)$ of the beta distribution can be set to model the proportional size of the daughter cell ($\frac{1}{2}$ for symmetric cell division; different values for asymmetric divisions, budding, and so on). The variance $\alpha \beta / ((\alpha+ \beta)^2 (\alpha + \beta + 1))$ of the beta distribution can be independently tuned to model the heterogeneity of the network distribution within the original (mother) cell.

We are then interested in the statistics

\begin{eqnarray}
  N & = & W_n + M_n + W_c + M_c \\
  h & = & (M_n + M_c) / N
\end{eqnarray}

of this model. The associated calculations are given in the Appendix. 

The mean and variance of $N$ can be readily derived using the laws of iterated expectation and total variance to account for the compound distributions of networked mtDNAs. As a ratio of random variables, analytic expressions for the mean and variance of $h$ are more challenging. We used a first-order Taylor expansion of $h$ to express $E(h)$ and $V(h)$ in terms of the (co)variances of the individual mtDNA counts, but found that this expression consistently underestimated numerical simulations of $V(h)$ (Fig. \ref{fig2}). We then asked whether the second-order Taylor expansion of $h$ would correct for this underestimation. While more complicated, the resulting expressions do better capture simulated $V(h)$ behaviour in many cases. For large $N_0$, and homogenous $U$, the dependence of the second-order expansion on higher moments introduces numerical issues that challenge its performance. A comparison of the two approaches to simulated data is shown in Fig. \ref{fig2}. Typically, the first-order approximation underestimates $V(h)$ when $p$ and $q$ are high (hence a high proportion of mtDNAs are embedded in the network), and the second-order approximation overcompensates, overestimating (though usually by a lower extent) $V(h)$ in this case. Including higher-order terms in the expansion will likely lead to eventually convergence, but at the cost of very complicated algebra, which we avoid here.

Several insights can be gained by considering particular cases. First, if $p = q - \delta$ and $E(u) = p_c$,

\begin{equation}
  V'(h) = \frac{1}{n} (1/p_c - 1) - \frac{1}{n p_c^2} V(u) (p - \delta(1-h)) - \frac{1}{p_c^2} V(u) h \delta^2
\end{equation}

It will be seen that there is a quadratic dependence of $V'(h)$ on $\delta$, the difference in wildtype and mutant allocation to the network. In the $\delta = 0$ case (hence wildtype and mutant mtDNAs are assigned equally to the network), this reduces to

\begin{equation}
  V'(h) = \frac{1}{n} (1/p_c - 1 - p V(u)/p_c^2)
\end{equation}

and if division is symmetric, $p_c = 1/2$ and

\begin{equation}
V'(h) = \frac{1}{n} (1 - 4 p V(u))
\end{equation}

This expression suggests that network structure provides a negative contribution to $V'(h)$  XXX this is why we have the blue streak. Does the second-order expression have a compensatory term we can quote?

To explore the case where mutant and wildtype network allocation differ, we set $p = 1, q = 0$, where for symmetric $p_c = 1/2$

\begin{eqnarray}
  V'(h) & = & 1/n + 4h V(u) (n-1)/n - 4h^2 V(u) \\
  & \simeq & 1/n + 4 h(1-h) V(u)
\end{eqnarray}

where the approximation holds for large $n$. Hence, unequal network allocation provides a heteroplasmy-dependent increase in heteroplasmy variance scaled by the variance of the inheritance network mass.

For mtDNA copy number, if $p = q - \delta$ and $E(u) = p_c$, 

\begin{equation}
  V(n) = p_c(1-p_c) n + (p - \delta h) V(u) n (n(p - \delta h) - 1)
\end{equation}

which if $\delta = 0$ reduces to

\begin{eqnarray}
  V(n) = p_c(1-p_c) n + n p (n p -1) V(u).
  & \simeq & p_c(1-p_c) n + n^2 p^2 V(u)
\end{eqnarray}

where the approximation is for the case $np \gg 1$, hence a high number of networked mtDNAs.

For symmetric divisions, $p_c = 1/2$ and

\begin{equation}
  V(n) = n/4 + n p (n p -1) V(u)
\end{equation}

then in the absence of a network, $p = 0$ and we recover the binomial $V(n) = n/4$, and if all mtDNAs are distributed in the network,

\begin{eqnarray}
  V(n) &=  &n/4 + V(u) (n-1) \\
  & \simeq& n/4 + n^2 V(u)
\end{eqnarray}

where the approximation holds for large mtDNA copy number.

In the absence of a network $p = 0, \delta = 0$, the expected binomial expressions are recovered:

\begin{eqnarray*}
  V'(h) & = & 1/(p_c n)  - 1/n \\
  V(n) & = & n p_c (1-p_c).
\end{eqnarray*}

Intuitively, these functional forms predict that $V'(h)$ is maximised for low $p_c$ (small daughter cells, where low copy number effects are most pronounced), falling to $1/n$ for symmetric $p_c = 1/2$, and towards zero for $p_c \rightarrow 1$ (as more and more of the original cell is retained, with correspondingly less variability in the inherited contents). The agreement between theory and simulation for different $p_c$ is shown in Figs. \ref{fig-asymm-map} and \ref{fig-asymm-moments} XXX better for some than others

\subsection*{MtDNA self-avoidance within the mitochondrial network allows variance reduction of copy number and heteroplasmy}
We next asked whether the variances of copy number $V(N)$ and heteroplasmy $V(h)$ could be reduced below their `null' binomial value through cellular processes. To this end, we modelled self-avoidance of mtDNA molecules within the network, reasoning that within-network sensing may allow controlled distribution, and hence a more even spread, of mtDNAs within the network. To accomplish this self-avoidance within our model, we enforce a `halo' of exclusion around each mtDNA placed within the network, so that another networked mtDNA cannot be placed within a distance $r$ of an existing one. The results are shown in Fig. \ref{fig-repel-map} and \ref{fig-both-moments}. 

Copy number variance $V(N)$ is decreased substantially by self-avoidance. In the case of a homogenous network and high proportions of mtDNA network inclusion, this decrease can extend below the binomial null model, allowing more faithful than binomial inheritance, as reported in yeast XXX. This sub-binomial inheritance requires both an even network distribution and mtDNA self-avoidance, and hence two levels of active cellular control. 

The case of maximum genetic bias in network inclusion ($p=0,q=1$ or $p=1,q=0$) is again informative. For a highly heterogenous network, this case again maximises $V(h)$. However, for an evenly spread network, this case instead gives minimum $V(h)$ values, even allowing variance to decrease below the binomial null case. XXX why? 

In the analytic model, a XXXThe even distribution of mtDNAs within the network can be captured by changing the model:
\begin{eqnarray*}
  W_n & = & p(1-h_0)N_0 U \\
  M_n & = & qh_0 N_0 U.
\end{eqnarray*}

However, this XXX

\subsection*{Diffusion of mtDNAs relaxes statistics towards the binomial case}
The mitochondrial network fragments before cell division. This fragmentation gives a time window during which mtDNAs that were previously constrained by the network structure can diffuse away from their initial position. In the limit of infinite diffusion, network structure will be forgotten and the mtDNA population will be randomly and uniformly distributed throughout the cell, leading to binomial inheritance patterns. We next investigated how lower amounts of diffusion away from the initial structure influence the patterns of mtDNA inheritance. To this end, we modelled XXX. The results are shown in Fig. \ref{fig-lambda}.

\begin{figure}
  \includegraphics[width=.7\textwidth]{plotlambda.png}
  \caption{Relaxation to binomial case by diffusion. $V(h)$ (vertical axis) against diffusion strength $\lambda$ (horizontal axis) for different repulsion values (columns) and numbers of seeds (rows). Each panel is a collection of different parameterisations.}
  \label{fig-lambda}
\end{figure}



\subsection*{Use of higher-order terms in the analytic approximation} 
Given the rather mixed performance of the second-order Taylor expansion at predicting trends in $V(h) $ in our physical model, we next investigated its performance for a purely statistical model. Here, we consider only draws of random variables from Eqns. Xxx, without explicitly simulating a network. We find that for heterogeneous $U$ and low $N_0$, the second-order approximation indeed refines the $V(h) $ estimates. However, for higher $N_0$, the increased magnitude of quantities involved challenges the numerical accuracy of the approximation, leading to diverging overcompensation with respect to the first-order result (as seen in some results above). We conclude that there is no simple recipe for deciding the `best' approximation to use in a given circumstance, though in most cases the first-order result with corrections suggested by the patterns in the second-order result would seem to apply. See Figs. \ref{fig-both-sim} and \ref{fig-both-moments} and \ref{fig-second-order}

\section*{Discussion} 
The function(s) of mitochondrial networks remain an open question XXX. In addition to xxx, These results suggest that network structure can be used as a means of controlling genetic population structure XXX

\newpage
\clearpage

\section*{Appendix}

\begin{figure}
  \includegraphics[width=\textwidth]{symm-repel-both-stats-repel.png}
  \caption{Partitioning stats for the repulsive case. Top left, $V(h)$. Top right, $V(N)$. Bottom left, $V(h)$ predictions from hypergeometric model. Bottom right, $V(N)$ predictions. The predictions can best be viewed as representing an extreme case which simulations may not reach.}
  \label{fig-repel-map}
\end{figure}

\begin{figure}
  \includegraphics[width=\textwidth]{symm-sim-theory-no-repel.png}
  \includegraphics[width=\textwidth]{symm-repel-sim-theory-repel.png}
  \caption{Theory-simulation comparison. First row, non-repulsive model, $V(h)$ for first- and second-order expansions, and $V(N)$. Second row, statistical simulations for some parameterisations of the non-repulsive model compared to analytic expressions. These good comparisons typically occur for low $N$ and variable $U$. Third row, repulsive model, $V(h)$ for first-order expansions (second-order is a placeholder, set to zero, without calculation), and $V(N)$. Second row, statistical simulations for some parameterisations of the non-repulsive model compared to analytic expressions. }
  \label{fig-both-sim}
\end{figure}

\begin{figure}
  \includegraphics[width=.5\textwidth]{symm-moments-no-repel.png} \\ \\
  \includegraphics[width=.5\textwidth]{symm-repel-moments-repel.png}
  \caption{Theory-sim comparison: individual moments and covariances. Top set: non-repulsive model. Bottom set: repulsive model.}
  \label{fig-both-moments}
\end{figure}

\begin{figure}
  \includegraphics[width=.5\textwidth]{symm-vh-stats-no-repel.png} \\ \\
  \caption{Theory-sim comparison for second-order approximation. Top row, simulated $V(h)$. Middle row, first-order expansion theory. Bottom row, second-order expansion.}
  \label{fig-second-order}
\end{figure}

%\begin{figure}
%  \includegraphics[width=.5\textwidth]{symm-vh-stats-no-repel.png} \\ \\
%    \includegraphics[width=.5\textwidth]{symm-repel-vh-stats-repel.png}  

%  \caption{Theory-sim comparison. Top set, non-repulsive model. Bottom set, repulsive model. Within each set: Top row, simulated $V(h)$. Middle row, first-order expansion theory. Bottom row, second-order expansion.}
%  \label{fig2}
%\end{figure}

\begin{figure}
  \includegraphics[width=\textwidth]{{asymm-0.25-both-stats-no-repel}.png}
  \caption{Partitioning stats for an asymmetric case where $ystar = 0.25$. Top left, $V(h)$. Top right, $V(N)$. Bottom left, $V(h)$ predictions. Bottom right, $V(N)$ predictions.}
  \label{fig-asymm-map}
\end{figure}

\begin{figure}
  \includegraphics[width=.5\textwidth]{{asymm-0.25-moments-no-repel}.png} \\ \\
    \includegraphics[width=.5\textwidth]{{asymm-0.80-moments-no-repel}.png} \\ \\
  \caption{Theory-sim comparison: individual moments and covariances. Top set: asymmetric $ystar = 0.25$. Bottom set: $ystar = 0.80$.}
  \label{fig-asymm-moments}
\end{figure}



\end{document}



  

